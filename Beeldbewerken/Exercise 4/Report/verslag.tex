\documentclass[10pt,a4paper]{article}
\usepackage[latin1]{inputenc}
\usepackage{amsmath}
\usepackage{amsfonts}
\usepackage{amssymb}
\usepackage{booktabs}
\usepackage{graphicx}
\usepackage{listings}
\usepackage{subfigure}
\usepackage{float}

\title{Local Structure - assignment 4}
\author{Jayke Meijer (6049885), Richard Torenvliet (6138861)}

\begin{document}
\maketitle

\section{Introduction}

In this exercise we implement a few exercises with the goal to learn 
the effect of Gaussian derivatives as part of a convolution. Using this,
we will finally implement the `Canny Edge Detector'.

\section{Analytical Local Structure}

To be done by Richard

\section{Gaussian Convolution}

To be done by Richard

\section{Separable Gaussian Convolution}

A performance increase for the Gaussian Convolution can be gained by
applying the convolution first in one direction and then in the
second.

\section{Gaussian Derivatives}



\section{Canny Edge Detector}

\end{document}